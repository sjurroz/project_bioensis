\documentclass{bmcart}

%%% Load packages
\usepackage{amsthm,amsmath}
\usepackage{graphicx}
%\RequirePackage[numbers]{natbib}
%\RequirePackage{hyperref}
\usepackage[utf8]{inputenc} %unicode support
%\usepackage[applemac]{inputenc} %applemac support if unicode package fails
%\usepackage[latin1]{inputenc} %UNIX support if unicode package fails
\usepackage{array}
\usepackage{float}
\usepackage{adjustbox}
\setlength\parskip{\baselineskip}
\usepackage{multirow}
\usepackage{hyperref}
\usepackage{xurl}
\usepackage{microtype}
\usepackage{textcomp}

\begin{document}

	\begin{frontmatter}
	
		\begin{fmbox}
			\dochead{Research}
		
			
			\title{Análisis de la Esclerosis Lateral Amiotrófica a través de la biología de sistemas}
			
			\author[
			  addressref={aff1},
			  email={santiagojuarroz@uma.es} ]{\inits{S.J.S.}\fnm{Santiago} \snm{Juarroz Surballe}} 
			\author[ 	
			  addressref={aff1},
			  email={gamy-@uma.es}
			]{\inits{G.M.Y.}\fnm{Gabriela Milenova} \snm{Yordanova}}
			\author[
			addressref={aff1},
			email={chediackmaga@uma.es}
			]{\inits{M.C.C.}\fnm{Maga C.} \snm{Chediack}}
			\author[ 	
			addressref={aff1},
			email={srozenblum@uma.es}
			]{\inits{S.R.}\fnm{Sebastián} \snm{Rozenblum}}
			

			
			\address[id=aff1]{%                           % unique id
			  \orgdiv{ETSI Informática},             % department, if any
			  \orgname{Universidad de Málaga},          % university, etc
			  \city{Málaga},                              % city
			  \cny{España}                                    % country
			}
		
		\end{fmbox}% comment this for two column layout
		
		\begin{abstractbox}
		
			\begin{abstract} % abstract
				La Esclerosis Lateral Amiotrófica (ELA) es una enfermedad neurodegenerativa compleja cuya etiología molecular exacta y heterogeneidad fenotípica presentan grandes desafíos para el diagnóstico y tratamiento. Este estudio aborda la patología desde la perspectiva de la biología de sistemas con el objetivo de dilucidar la arquitectura modular de sus genes implicados y sus mecanismos biológicos subyacentes. Se construyeron y analizaron redes de interacción proteína-proteína (PPI) óptimas, aplicando algoritmos de detección de comunidades y análisis de enriquecimiento funcional mediante las bases de datos Gene Ontology (GO), KEGG y REACTOME.
				
				Los resultados revelaron la existencia de módulos funcionales discretos y altamente interconectados. Se identificó un agrupamiento clave asociado a la neuroinflamación y comunicación intercelular (genes ERBB4, PSEN1, TREM2), destacando la vía de señalización DAP12 y la proteólisis intramembrana. Asimismo, se caracterizó un clúster central ("MAYOR") que integra procesos de autofagia selectiva, transporte y procesamiento de ARNm, y tráfico vesicular, evidenciando una falla sistémica en el reciclaje celular y la dinámica del citoesqueleto. El análisis comparativo sugiere una fuerte convergencia molecular con las enfermedades de Alzheimer y Huntington, indicando que la ELA comparte vías críticas de neurodegeneración e inmunidad innata. Esta red de interacciones ofrece una visión integral de la patología, sugiriendo que el daño no es solo intrínseco a la motoneurona, sino derivado de una comunicación defectuosa con su microambiente.
				
			\end{abstract}
			
			\begin{keyword}
			\kwd{ELA}
			\kwd{gen}
			\kwd{interacción}
			\kwd{ontología}
			\kwd{Redes PPI}
			\kwd{Biología de sistemas}
			\kwd{Enriquecimiento Funcional}
			
			\end{keyword}
		
		
		\end{abstractbox}
	
	\end{frontmatter}
	
	\section{Introducción}
Se estima que más del 70\% de las muertes en el mundo están relacionadas con enfermedades no transmisibles, es decir, patologías de larga duración y progresión lenta que no se transmiten entre personas y que incluyen afecciones como el cáncer, las enfermedades cardiovasculares o las neurodegenerativas \cite{who2023}. Dentro de este último grupo, patologías como la esclerosis lateral amiotrófica (ELA) suponen un desafío particular, tanto por la complejidad de su diagnóstico como por la evolución impredecible de sus síntomas \cite{Hardiman2017}. En este contexto, la detección temprana supone una ventaja y, al mismo tiempo, un desafío crucial para mitigar el impacto clínico en los pacientes y optimizar los sistemas de salud. 

	
\section{Materiales y métodos}
Para este trabajo se realizó el análisis de la red de interacción proteína-proteína (PPI, protein-protein interaction) del fenotipo HP:0007354 (esclerosis lateral amiotrófica), en donde se utilizaron diferentes herramientas que posibilitaron el estudio de la red. Entre ellas se destacan: 
\begin{itemize}
	\item \textbf{HPO}: definida por la Human Gene Ontology organization como un vocabulario estandarizado que utilizan médicos e investigadores para describir de manera uniforme y precisa las anomalías fenotípicas observadas en las enfermedades humanas.
	
	\item \textbf{Python}: lenguaje de programación interpretado. Es una herramienta dominante para el parsing (análisis sintáctico) de archivos biológicos complejos y manipulación de grandes volúmenes de datos tabulares mediante bibliotecas como Pandas y Numpy. Que a su vez permite construir y gestionar pipelines (flujos de trabajo) que automatizan análisis complejos.
	
	\item \textbf{STRINGdb} (Search Tool for the Retrieval of Interacting Genes/protein) es una base de datos biológica y un servidor web de acceso libre enfocado principalmente en las Interacciones proteína-proteína (PPIs). Esta herramienta es muy útil para integrar las asociaciones funcionales entre las proteínas de un organismo.
\end{itemize}


La implementación se realizó en Python, debido a su versatilidad, simplicidad en la sintaxis y amplia disponibilidad de bibliotecas especializadas, que facilitan tanto la comunicación con las APIs de las bases de datos más utilizadas como la representación gráfica de redes complejas. Específicamente, se utilizaron las siguientes librerías:

\begin{itemize}
	\item \textbf{Numpy (v. 1.26.4)}: librería esencial para cálculos numéricos y operaciones matriciales.
	
	\item \textbf{Matplotlib (v. 3.9.2)}: librería para visualizar datos e imágenes y generar gráficas.
	
	\item \textbf{Pandas (v. 2.2.3)}: librería para manipular, limpiar y analizar datos en forma de tablas (DataFrames).
	
	\item \textbf{Requests (v. 2.32.3)}: librería para realizar solicitudes HTTP y comunicarse con APIs o servicios web.
	
	\item \textbf{Json}: módulo para leer, escribir y procesar datos en formato JSON. Es parte de la biblioteca estándar de Python, por lo que no requiere instalación ni tiene versión propia.
	
	\item \textbf{igraph}: librería para crear, analizar y visualizar redes o grafos, con soporte para métricas y tareas adicionales.
\end{itemize}

El flujo de trabajo diseñado consta de cinco etapas principales que abarcan desde la obtención de genes relacionados con el fenotipo de interés hasta la extracción de conocimiento biológico de la red PPI. Para algunas de las etapas se realizaron múltiples ejecuciones, variando en cada una los parámetros principales. De esta forma, se obtuvieron múltiples combinaciones de resultados, con el objetivo de evaluar qué configuración presentaba los resultados más favorables. Un esquema conceptual del flujo de trabajo puede verse en la Figura \ref{fig:diagramadeflujobiosis}.

En primer lugar, se recopiló la lista de genes asociados al fenotipo de interés. Para hacerlo, se adoptaron dos enfoques distintos: por un lado, se utilizaron los genes encontrados tras una revisión bibliográfica manua l (Tabla 1); por el otro, se extrajeron mediante una función de STRINGdb a partir de la API de la HPO.

A continuación, mediante la lista de genes, se generó la red PPI para cada una de las dos listas de genes. En este punto propusimos tres posibles umbrales de puntuación para dar lugar a la red: uno laxo, con un valor de 300; otro intermedio, de 700; y el más estricto, igual a 900. Esto permitió comparar cómo varía la estructura de la red en función del nivel de confianza asignado a las interacciones, evaluando si los patrones observados se mantienen consistentes al aumentar el umbral de evidencia requerido. Una vez obtenidas las seis redes, se representan gráficamente mediante la librería Igraph. Para cada una de ellas, se llevó a cabo un pequeño análisis exploratorio inicial, en el cual se midieron distintas propiedades básicas de topología como número de nodos, grado, densidad, centralidad, modularidad y dispersión.

\begin{figure}[!h]
	\centering
	\includegraphics[width=0.7\linewidth]{figures/Diagrama_de_flujo_BioSis.png}
	\caption{Diagrama de trabajo paso a paso}
	\label{fig:diagramadeflujobiosis}
\end{figure}
\newpage

Luego, resultó necesario entender los posibles agrupamientos o clusters que tuvo cada red. Este proceso se llevó a cabo con el objetivo de descubrir grupos de genes altamente conectados entre sí, los cuales podrían corresponder a funciones biológicas muy relacionadas con la ELA. Nuevamente, se propusieron dos estrategias diferentes de agrupamiento, que fueron elegidos por su eficacia en conjuntos pequeños de datos (menos de 150 genes) y  su interpretabilidad clara. \ref{Hardiman2023}

\subsubsection*{Clustering basado en la centralidad de interediacion de enlace (Edge Betweenness Clustering)}

Es un método de detección de comunidades propuesto por Girvan y Newman en el año 2002. Este algoritmo consiste en calcular la centralidad de intermediación para todas las aristas donde se eliminan aquellas con los valores más altos, dicho proceso se hace de manera iterativa con el objetivo de separar la red en subconjuntos mayormente conectados de forma interna.

El proceso termina al obtener la red con agrupamientos que presenten el valor más alto de la modularidad  métrica, la cual cuantifica la calidad de la conexión de los agrupamientos con el resto de la red. Dicha métrica tiene un intervalo [-1,1] donde el valor más cercano o igual a 1 determina que el  agrupamiento está bien definido. \ref{newman2004modularity}

\textbf{Significado de los valores de la modularidad (Q)}
\begin{itemize}
	\item \textbf{$Q\approx 1$}  Agrupamiento fuerte y bien definido
	\item \textbf{$0 < Q < 1$} Partición clara aunque no completamente aislada ya que hay cierta conectividad entre los agrupamientos.
	\item \textbf{$Q\approx 0$}  Estructura compuesta por conexiones aleatorias.
	\item \textbf{$-1 < Q < 0$} La partición es peor que el azar, ya que hay más conexiones entre comunidades que dentro de ellas.
	\item \textbf{$Q\approx -1$} Agrupamiento totalmente incoherente
\end{itemize}

\subsubsection*{Clustering basado en el algoritmo InfoMap}
Es un método de detección de comunidades propuesto por Martin Rosvall y Carl T.Bergstrom en el año 2008 que no depende de una métrica estructural predeterminada y detecta comunidades de tamaño variable con submódulos jerárquicos.Este algoritmo detecta los agrupamientos analizando el flujo de información entre nodos usando caminatas aleatorias que simulan el movimiento dentro de una red y eligen el próximo nodo al azar.vspace{1em}

Las múltiples caminatas se utilizan para calcular la cantidad de bits que se necesitan para describir el recorrido, si el recorrido pasó una gran parte dentro de un mismo grupo de nodos se considera una comunidad densamente conectada internamente mientras que, en el caso contrario se considera que hay una conexión más débil entre comunidades.  \ref{rosvall2008infomap}

	
\section{Resultados}
A partir de la aplicación de la metodología, para el caso de la lista creada a partir del HPO obtuvimos los datos que se muestran en la tabla 1. Mientras que para la lista manual se obtuvieron los datos que se muestran en la tabla 2. 

\begin{table}[h!]
	
	\centering
	
	\renewcommand{\arraystretch}{1.5} % Aumenta el espacio entre filas para que se lea mejor
	
	\setlength{\tabcolsep}{8pt}       % Aumenta el espacio entre columnas
	
	\begin{tabular}{|c|c|c|c|}
		
		\hline
		
		\textbf{ID} & \textbf{\textit{Edge Betweenness}} & \textbf{\textit{Fast Greedy}} & \textbf{\textit{InfoMap}} \\ \hline
		
		\textbf{300} & (n\_cluster)/(categorías) & (n\_cluster)/(categorías) & (n\_cluster)/(categorías) \\ \hline
		
		\textbf{600} & (n\_cluster)/(categorías) & (n\_cluster)/(categorías) & (n\_cluster)/(categorías) \\ \hline
		
		\textbf{900} & (n\_cluster)/(categorías) & (n\_cluster)/(categorías) & (n\_cluster)/(categorías) \\ \hline
		
	\end{tabular}
	
	
	\caption{Resultados de agrupamiento por umbral y algoritmo.}
	
	\label{tab:resultados_cluster}
	
\end{table}

Como se muestra en las tablas 1 y 2, el análisis de las redes mostró un balance óptimo al utilizar el score de 700. En contraste el umbral de 300 generó una red con una estructura laxa, mientras que el score 900 resultó estricto, limitando las interacciones. 

En la figura 1.a y 1.b se ilustra el rendimiento de los  métodos de agrupamiento utilizados. En ella se revela que el algoritmo de Edge Betweenness ofreció los resultados más favorables, tanto para la red generada manualmente como para la basada en HPO.

	\section{Discusión}

Una vez obtenidos los resultados, el objetivo principa del estudio fue dilucidar la arquitectura modular de los genes implicados en la Esclerosis Lateral Amiotrófica (ELA), a partir de una búsqueda exhaustiva de las redes funcionales con mecanismos biológicos subyacentes. 
Para ello se implementó un flujo de trabajo comparativo variando las fuentes de datos (HPO vs Manual), los umbrales de confianza “score” y los algoritmos de detección de comunidades. A continuación, se discuten las decisiones metodológicas y los hallazgos biológicos derivados de este análisis jerárquico. 

\subsection{Impacto del umbral en la topología de la red}

La primera etapa del análisis consistió en determinar el umbral de interacción óptimo en STRING. La selección del \textit{score} no es trivial, ya que determina una alta similitud entre nodos de la red. 

Nuestros resultados indicaron que el \textit{umbral} de 300 generó redes excesivamente densas, generando muchos agrupamientos pequeños y aislados, perdiendo conexiones entre módulos funcionales y dificultando la identificación de módulos discretos. Este fenómeno es consistente con lo reportado por \cite{Szklarczyk2023}, quienes advierten que los porcentajes bajos en STRINGdb introducen falsos positivos que oscurecen la estructural modular real. 

Por el contrario, el \textit{umbral} de 900 resultó en una red fragmentada y dispersa. Si bien las interacciones poseen alta evidencia, esta restricción provocó la pérdida de conectividad entre módulos funcionales y el aislamiento de genes potencialmente relevantes generando falsos negativos, limitando la visión sistémica de la patología.

En consecuencia, el \textit{umbral} de 700 emergió como el punto de equilibrio óptimo. Este umbral permitió conservar una estructura topológica rica y conectada, eliminando el ruido de fondo pero preservando las interacciones biológicamente significativas necesarias para la detección de comunidades. Esta estrategia es respalada por estudios recientes en biologia de sistemas \cite{Menche2015, Cheng2019}.

\subsubsection{La discrepancia funcional de InfoMap}
Un hallazgo particular fue el comportamiento del algoritmo \textit{InfoMap} aplicado a la lista manual con \textit{score} 700. Aunque topológicamente generó particiones con alta modularidad, el análisis de enriquecimiento funcional (GO) demostró que estos clústeres carecían de relevancia biológica o coherencia interna. Esto se alinea con el concepto de “sesgo de inspección” discutido por \cite{Stoeger2019}.

Esta discrepancia puede explicarse por el "sesgo de estudio" (\textit{study bias}) inherente a la lista manual discutida por \cite{Stoeger2019}. \textit{InfoMap} basa su partición en el flujo de información. Es posible que en la lista manual, los genes estén conectados por patrones de citación en la literatura, es decir, genes que frecuentemente  se estudian juntos  más que por una afinidad biológica real. \textit{InfoMap} detectó este flujo, creando grupos matemáticamente válidos pero biológicamente no. Esto justificó su descarte en favor de \textit{Edge Betweenness}, que prioriza la estructura modular física de la red.



	\section{Conclusiones}
Este estudio  ha permitido el análisis de redes de interacción proteína-proteína en el contexto de la Esclerosis Lateral Amiotrófica (ELA) a partir de la implementación la biología de sistemas para deconstruir la enfermedad genética del fenotipo . Nuestros resultados demuestran que la integración de datos fenotípicos estandarizados (HPO) con un (\textit{score} 700) y el algoritmo de \textit{Edge Betweenness} constituye la estrategia más adecuada para el análisis del enriquecimiento funcional de las redes PPI compuestas por genes involucrados en nuestro fenotipo.

Desde una perspectiva metodológica, se concluye que:
\begin{itemize}
	\item La selección del umbral es crítica: un \textit{score} de 700 evita tanto la saturación de ruido (falsos positivos) observada en umbrales bajos como la pérdida de información funcional (falsos negativos) de los umbrales estrictos.
	\item La curación manual de genes, aunque es valiosa, presenta un "sesgo de inspección" significativo que puede inducir a errores en algoritmos basados en flujo como \textit{InfoMap}. En contraste, el uso de la ontología HPO demostró ser superior para capturar una red más rica y funcionalmente coherente.
\end{itemize}

En el ámbito biológico, la validación de la red priorizada confirmó la centralidad de genes canónicos como \textit{SOD1}, \textit{C9orf72}, \textit{TARDBP} y \textit{FUS} dentro del módulo principal, ratificando la capacidad del método para recuperar mecanismos patogénicos conocidos. Asimismo, el análisis topológico permitió discriminar eficazmente genes periféricos como \textbf{PON1} y \textbf{PON3}, cuya desconexión del componente mayoritario sugiere un rol secundario en la red de interacción directa.
Futuras investigaciones deberán centrarse en la validación experimental de las interacciones clave dentro de este módulo, así como en la integración de datos multi-ómicos (transcriptómicos, proteómicos) para modelar la dinámica de esta red a lo largo de la progresión de la enfermedad. En definitiva, este estudio sienta las bases para una exploración más rigurosa y sistémica de la ELA, orientando la búsqueda de dianas terapéuticas hacia los nodos críticos que sostienen esta arquitectura patogénica.


	
	\begin{backmatter}

		
		\bibliographystyle{bmc-mathphys} 
		\bibliography{bibliography}     
	
	\end{backmatter}
\end{document}
