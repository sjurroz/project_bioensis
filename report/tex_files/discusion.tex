\section{Discusión}

Una vez obtenidos los resultados, el objetivo principal del estudio fue dilucidar la arquitectura modular de los genes implicados en la Esclerosis Lateral Amiotrófica (ELA), a partir de una búsqueda exhaustiva de las redes funcionales con mecanismos biológicos subyacentes. 
Para ello se implementó un flujo de trabajo comparativo variando las fuentes de datos (HPO vs Manual), los umbrales de confianza "score", los algoritmos de detección de comunidades y por último, el análisis del enriquecimiento funcional. A continuación, se discuten las decisiones metodológicas y los hallazgos biológicos derivados de este análisis jerárquico. 

\subsection{Impacto del umbral en la topología de la red}

La primera etapa del análisis consistió en determinar el umbral de interacción óptimo en STRING. La selección del \textit{score} no es trivial, ya que determina una alta similitud entre nodos de la red. 

Nuestros resultados indicaron que el \textit{umbral} de 300 generó redes excesivamente densas, generando muchos agrupamientos pequeños y aislados, perdiendo conexiones entre módulos funcionales y dificultando la identificación de módulos discretos. Este fenómeno es consistente con lo reportado por \cite{Szklarczyk2023}, quienes advierten que los porcentajes bajos en STRINGdb introducen falsos positivos que oscurecen la estructural modular real. 

Por el contrario, el \textit{umbral} de 900 resultó en una red fragmentada y dispersa. Si bien las interacciones poseen alta evidencia, esta restricción provocó la pérdida de conectividad entre módulos funcionales y el aislamiento de genes potencialmente relevantes generando falsos negativos, limitando la visión sistémica de la patología.

En consecuencia, el \textit{umbral} de 700 emergió como el punto de equilibrio óptimo. Este umbral permitió conservar una estructura topológica rica y conectada, eliminando el ruido de fondo pero preservando las interacciones biológicamente significativas necesarias para la detección de comunidades. Esta estrategia es respalada por estudios recientes en biología de sistemas \cite{Cheng2011}.

\subsubsection{La discrepancia funcional de InfoMap}
Un hallazgo particular fue el comportamiento del algoritmo \textit{InfoMap} aplicado a la lista manual con \textit{score} 700. Aunque topológicamente generó particiones con alta modularidad, el análisis de enriquecimiento funcional (GO) demostró que estos clústeres carecían de relevancia biológica o coherencia interna. Esto se alinea con el concepto de "sesgo de inspección" discutido por \cite{Stoeger2019}.

 Infomap basa su partición en el flujo de información. Es posible que en la lista manual, los genes estén conectados por patrones de citación en la literatura, es decir, genes que frecuentemente  se estudian juntos  más que por una afinidad biológica real. \textit{InfoMap} detectó este flujo, creando grupos matemáticamente válidos pero biológicamente no. Esto justificó su descarte en favor de \textit{Edge Betweenness}, que prioriza la estructura modular física de la red.

\subsection{Análisis biológico y caracterización de los módulos}

Al profundizar en los resultados obtenidos con la configuración óptima (\textit{Edge Betweenness} + \textit{Score} 700), se observó una consistencia notable entre las redes generadas de forma manual y automáticamente (HPO), aunque con matices importantes.

En ambas redes se identificó sistemáticamente un clúster pequeño y aislado compuesto por los genes \textbf{PON1} y \textbf{PON3}. A pesar de su recurrencia, el análisis de enriquecimiento funcional mediante Gene Ontology (GO) reveló que estos genes están asociados principalmente al catabolismo de lactonas y actividad arilesterasa. Su aislamiento topológico en la red, como se observa en las Figuras 5 y 6, sugiere que su presencia se debe a una alta similitud de secuencia o co-expresión, pero carecen de una significancia biológica central para los mecanismos patogénicos primarios de la ELA analizados en este contexto. Esta interpretación se apoya en el estudio de \cite{Gagliardi2019}, quienes investigaron específicamente los polimorfismos de PON1 en pacientes con ELA y concluyeron que, aunque existe una posible asociación, su papel en la patogénesis de la enfermedad es periférico y no central en los mecanismos neurodegenerativos principales. Por tanto, se consideraron un hallazgo periférico y no prioritario.

Por otro lado, el clúster principal mostró diferencias interesantes entre las fuentes de datos:
\begin{itemize}
	\item La red basada en \textbf{HPO} generó un módulo principal más extenso (14 genes) en comparación con la lista manual (12 genes).
	\item Aunque la composición génica es muy similar, la red HPO logró capturar una mayor riqueza de interacciones, lo que se tradujo en una anotación funcional más completa.
\end{itemize}
Esto sugiere que, si bien la curación manual es precisa, la extracción automática basada en fenotipos (HPO) es capaz de recuperar interacciones sutiles que enriquecen el contexto biológico. 
El cluster principal "MAYOR" que podemos observar en las figuras 3 y 5, agrupa a los genes más emblemáticos de la ELA, como C9orf72, SOD1, TARDBP y FUS. La co-aparición de estos genes valida nuestro enfoque, ya que SOD1 fue el primer gen identificado en las formas familiares de la enfermedad y sus mutaciones son un pilar etiológico relevante.

En resumen, la combinación de un \textit{score} de alta confianza (700) con el algoritmo \textit{Edge Betweenness} sobre datos derivados de HPO constituyó la estrategia más eficaz. Esta configuración permite filtrar el ruido, superar los sesgos de la literatura manual y aislar módulos funcionales coherentes, descartando agrupamientos espurios (como los de PON1/PON3) y centrando el análisis en los procesos biológicos nucleares de la esclerosis lateral amiotrófica.

Por consiguiente se seleccionó la red óptima para realizar un análisis más profundo del enriquecimiento funcional de esta misma, partiendo de ya haber hecho una mención de grandes rasgos preliminares de este análisis. Para esto se utilizó dos herramientas bioinformáticas complementarias a la ya mencionada (GO) que son KEGG y REACTOME, las cuales se usaron para analizar la red PPI como un todo. REACTOME analizó en qué proceso biológico se ve implicado el \textit{cluster}, mientras que GO amplía con la ubicación y la función de dicho proceso y por último KEGG indicó el contexto clínico. Como se mencionó anteriormente uno de los tres agrupamientos no presentó relevancia alguna en el contexto del fenotipo siendo descartado para continuar en el análisis. 

El análisis del \textit{cluster} conformado por los genes ERBB4, PSEN1 y TREM2 reveló su implicación predominante en mecanismos de neuroinflamación y comunicación intercelular, desviando el foco desde los fallos intrínsecos de la motoneurona hacia la transducción de señales y la inmunidad innata. REACTOME evidenció una convergencia en la señalización mediada por receptores de superficie, destacando específicamente la vía de señalización DAP12 (impulsada por TREM2), la cual es crucial para la activación microglial y la respuesta inflamatoria del cerebro. Complementariamente, GO resaltó el proceso de proteólisis intramembrana (regulada por PSEN1), ubicando fuertemente a estos componentes en la membrana plasmática y el sistema endomembranoso. Esto sugiere que la patología en este módulo se desarrolla en la superficie celular, ejecutando programas de migración y activación de células mieloides. Finalmente, KEGG indicó que este perfil molecular mimetiza vías observadas en la enfermedad de Alzheimer, lo que debe interpretarse como la identificación de mecanismos de 'daño colateral' o vías de riesgo compartidas entre ambas neurodegeneraciones.

Respecto al agrupamiento denominado 'MAYOR',  captura la arquitectura molecular central de la Esclerosis Lateral Amiotrófica, donde se distinguieron tres módulos funcionales muy  vinculados al fenotipo. El primero, fundamental para la eliminación de agregados citotóxicos, se relaciona con procesos de autofagia selectiva y degradación proteica (SQSTM1, OPTN, UBQLN2, VCP). El segundo módulo abarca el transporte y procesamiento de ARNm intracelular (FUS, TARDBP, SETX). El tercero implica defectos en el transporte vesicular (ALS2, CHMP2B, FIG4), reflejando la falla sistémica en el reciclaje celular y la dinámica del citoesqueleto típica de la enfermedad. La interpretación mediante GO sugiere una disfunción en la organización del citoesqueleto acoplada a una activación del sistema inmune innato, lo que implica una comunicación defectuosa con el microambiente. Por su parte, KEGG sitúa a este agrupamiento principalmente en el mapa de la ELA, aunque destaca una fuerte superposición con las enfermedades de Alzheimer y Huntington. Esto resulta lógico dado que los fallos en el transporte axonal y la función lisosomal son mecanismos neurodegenerativos transversales. De esta forma, se define un módulo funcional integral de degradación de patógenos y residuos, vinculando la neurodegeneración con la inmunidad innata.


