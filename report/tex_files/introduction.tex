\section{Introducción}
Se estima que más del 70\% de las muertes en el mundo están relacionadas con enfermedades no transmisibles, es decir, patologías de larga duración y progresión lenta que no se transmiten entre personas y que incluyen afecciones como el cáncer, las enfermedades cardiovasculares o las neurodegenerativas \cite{who2023}. Dentro de este último grupo, patologías como la esclerosis lateral amiotrófica (ELA) suponen un desafío particular, tanto por la complejidad de su diagnóstico como por la evolución impredecible de sus síntomas \cite{Hardiman2017}. En este contexto, la detección temprana supone una ventaja y, al mismo tiempo, un desafío crucial para mitigar el impacto clínico en los pacientes y optimizar los sistemas de salud. 
