\section{Introducción}
Se estima que más del 70\% de las muertes en el mundo están relacionadas con enfermedades no transmisibles, es decir, patologías de larga duración y progresión lenta que no se transmiten entre personas y que incluyen afecciones como el cáncer, las enfermedades cardiovasculares o las neurodegenerativas \cite{who2023}. Dentro de este último grupo, patologías como la esclerosis lateral amiotrófica (ELA) suponen un desafío particular, tanto por la complejidad de su diagnóstico como por la evolución impredecible de sus síntomas \cite{Hardiman2017}. En este contexto, la detección temprana supone una ventaja y, al mismo tiempo, un desafío crucial para mitigar el impacto clínico en los pacientes y optimizar los sistemas de salud. 

A día de hoy, los avances tecnológicos en el área de la biología de sistemas, abarcando campos más específicos como la genómica, la proteómica y la metabolómica, han demostrado producir resultados significativos en el diagnóstico y la detección de numerosas patologías \cite{Kitano2002}. Estos enfoques han sido aplicados con éxito en el análisis de enfermedades complejas como el cáncer \cite{Barabasi2011}, la diabetes tipo 2 \cite{Nielsen2017} y las enfermedades neurodegenerativas \cite{Tian2022}, permitiendo identificar biomarcadores, redes moleculares disfuncionales y posibles dianas terapéuticas.

Para interpretar la gran cantidad de datos genómicos y vincularlos a manifestaciones clínicas, es fundamental utilizar vocabularios estandarizados. En este contexto, el sistema HPO (Human Phenotype Ontology) \cite{hpo2025} describe tanto signos como síntomas y sus características clínicas asociadas a las enfermedades humanas. La ELA, definida por un conjunto característico de síntomas, se representa en HPO como un fenotipo complejo bajo el identificador HP:0007354, el cual se asocia a un conjunto específico de genes.

La ELA es una enfermedad neurodegenerativa que se caracteriza por una degeneración de neuronas motoras, tanto a nivel superior como a nivel inferior. Esta patología tiene una tasa anual de diagnóstico a nivel mundial de entre 2 y 11 casos por cada 100.000 habitantes. Específicamente, en Europa la incidencia anual es de 2.16 casos por cada 100.000 habitantes. Esta afecta en mayor medida a los hombres (3 por cada 100.000 habitantes), por sobre las mujeres (2.4 por cada 100.00 habitantes) \cite{Beghi_2010}.
Si bien aún se desconocen sus causas, se ha logrado entender su patogénesis, principalmente relacionadas con mutaciones cromosómicas [cita 5]. La mayoría de estas mutaciones son de fenotipo único, excepto por la ELA tipo 1 que es causada por una mutación en el gen SOD1 que codifica para la superóxido dismutasa 1, una proteína que, al plegarse de forma incorrecta, genera toxicidad.

Los estudios muestran que, en el 75\% de los pacientes, la enfermedad comienza de forma focal, distal y asimétrica seguida por una progresión anatómica lógica a grupos contiguos de motoneuronas \cite {rowland2001}. Usualmente, es una enfermedad de aparición esporádica, pero el 10\% de los casos puede ser de carácter familiar, con una herencia autosómica dominante [CITAR].  El patrón de herencia es, en la mayoría de las variantes, aunque hay casos de herencia autosómica recesiva (AR) que aparecen en las formas juveniles de ELA. 
Respecto a la ELA, se han detectado más de 20 genes implicados en la aparición de esta patología (en la reciente actualización de cartera de servicios genómica, presentada el 23 de enero de 2024 por el Ministerio de Sanidad y Consumo), aunque son las mutaciones de cuatro genes (SOD1, TARDBP, FUS y C9orf72) las que causan más del 50\% de los casos familiares \cite {dewan2021}. 

Los tratamientos actuales se basan en disminuir la progresión de la enfermedad y tratar los síntomas. Desde el punto de vista farmacológico solo existe un tratamiento aprobado, el cual se denomina Riluzole, el cual permite a los pacientes mejorar su calidad de vida.[CITA]
La ausencia de una cura subraya la importancia de los tratamientos actuales, cuyo objetivo es ralentizar la progresión de la enfermedad y manejar los síntomas para mejorar la calidad de vida de los pacientes. La ELA es diagnosticada mediante criterios clínicos basados en la sintomatología, ya que no existen biomarcadores definitivos para la enfermedad. Esta limitación retrasa el diagnóstico y la intervención temprana. 


