\section{Introducción}
Se estima que más del 70\% de las muertes en el mundo están relacionadas con enfermedades no transmisibles, es decir, patologías de larga duración y progresión lenta que no se transmiten entre personas y que incluyen afecciones como el cáncer, las enfermedades cardiovasculares o las neurodegenerativas \cite{who2023}. Dentro de este último grupo, patologías como la esclerosis lateral amiotrófica (ELA) suponen un desafío particular, tanto por la complejidad de su diagnóstico como por la evolución impredecible de sus síntomas \cite{Hardiman2017}. En este contexto, la detección temprana supone una ventaja y, al mismo tiempo, un desafío crucial para mitigar el impacto clínico en los pacientes y optimizar los sistemas de salud. 

A día de hoy, los avances tecnológicos en el área de la biología de sistemas, abarcando campos más específicos como la genómica, la proteómica y la metabolómica, han demostrado producir resultados significativos en el diagnóstico y la detección de numerosas patologías \cite{Kitano2002}. Estos enfoques han sido aplicados con éxito en el análisis de enfermedades complejas como el cáncer \cite{Barabasi2011}, la diabetes tipo 2 \cite{Nielsen2017} y las enfermedades neurodegenerativas \cite{Tian2022}, permitiendo identificar biomarcadores, redes moleculares disfuncionales y posibles dianas terapéuticas.

Para interpretar la gran cantidad de datos genómicos y vincularlos a manifestaciones clínicas, es fundamental utilizar vocabularios estandarizados. En este contexto, el sistema HPO (Human Phenotype Ontology) \cite{hpo2025} describe tanto signos como síntomas y sus características clínicas asociadas a las enfermedades humanas. La ELA, definida por un conjunto característico de síntomas, se representa en HPO como un fenotipo complejo bajo el identificador HP:0007354, el cual se asocia a un conjunto específico de genes.

