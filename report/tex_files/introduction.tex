\section{Introducción}
Se estima que más del 70\% de las muertes en el mundo están relacionadas con enfermedades no transmisibles, es decir, patologías de larga duración y progresión lenta que no se transmiten entre personas y que incluyen afecciones como el cáncer, las enfermedades cardiovasculares o las neurodegenerativas \cite{who2023}. Dentro de este último grupo, patologías como la esclerosis lateral amiotrófica (ELA) suponen un desafío particular, tanto por la complejidad de su diagnóstico como por la evolución impredecible de sus síntomas \cite{Hardiman2017}. En este contexto, la detección temprana supone una ventaja y, al mismo tiempo, un desafío crucial para mitigar el impacto clínico en los pacientes y optimizar los sistemas de salud. 

A día de hoy, los avances tecnológicos en el área de la biología de sistemas, abarcando campos más específicos como la genómica, la proteómica y la metabolómica, han demostrado producir resultados significativos en el diagnóstico y la detección de numerosas patologías \cite{Kitano2002}. Estos enfoques han sido aplicados con éxito en el análisis de enfermedades complejas como el cáncer \cite{Barabasi2011}, la diabetes tipo 2 \cite{Nielsen2017} y las enfermedades neurodegenerativas \cite{Tian2022}, permitiendo identificar biomarcadores, redes moleculares disfuncionales y posibles dianas terapéuticas. 

Para interpretar la gran cantidad de datos genómicos y vincularlos a manifestaciones clínicas, es fundamental utilizar vocabularios estandarizados. En este marco, el sistema Human Phenotype Ontology (HPO) \cite{hpo2025} describe tanto signos como síntomas y sus características clínicas asociadas a las enfermedades humanas. La ELA, definida por un conjunto característico de síntomas, se representa en la HPO como un fenotipo específico bajo el identificador \textit{HP:0007354}, el cual se asocia a un grupo determinado de genes \cite{ELA_HPO}.

La ELA se caracteriza por una degeneración de las neuronas motoras, tanto a nivel superior como a nivel inferior \cite{xie2023cortical}. Esta patología tiene una tasa anual de diagnóstico a nivel mundial de entre 2 y 11 casos por cada 100.000 habitantes \cite{Beghi2010}. Específicamente, en Europa la incidencia anual es de 2.16 casos por cada 100.000 habitantes  \cite{Beghi2010}. Esta afecta en mayor medida a los hombres (3 por cada 100.000 habitantes), por sobre las mujeres (2.4 por cada 100.00 habitantes) \cite{Beghi2010}. La enfermedad es mayoritariamente esporádica, si bien aproximadamente el 5-10\% de los casos se presenta en formas familiares con herencia autosómica dominante \cite{Barberio2023, Kiernan2021}. En las formas juveniles de ELA también se han descrito mutaciones con herencia autosómica recesiva, siendo estas menos frecuentes pero clínicamente relevantes \cite{Wu2016}.

Si bien todavía no se conocen completamente las causas de la ELA, se ha avanzado sustancialmente en comprender su patogénesis y base genética. Se han identificado mutaciones en múltiples genes —entre ellos \textit{SOD1}, \textit{C9orf72}, \textit{TARDBP}, \textit{FUS}, \textit{NEK1}, \textit{OPTN}, \textit{TBK1}, entre otros— que pueden causar o aumentar el riesgo de ELA, tanto en formas familiares como, en algunos casos, esporádicas \cite{VanDaele2023}. En particular, las mutaciones en \textit{SOD1} fueron de las primeras identificadas y siguen siendo relevantes: estos cambios genéticos pueden alterar la superóxido dismutasa 1, una proteína que, al plegarse de forma incorrecta, genera toxicidad mediante mecanismos de disfunción mitocondrial \cite{Gagliardi2023}. Además, los fenotipos asociados a mutaciones en este gen son muy variables \cite{Berdynski2025}. Se estima que mutaciones en cuatro genes —\textit{SOD1}, \textit{TARDBP}, \textit{FUS} y \textit{C9orf72}— explican más del 50\% de los casos familiares de ELA \cite{Rummens2025}. No obstante, el número de genes implicados continúa aumentando conforme se producen avances en la investigación.

En la  \autoref{tab:genes_introduccion} se presenta un resumen de la clasificación funcional de los principales genes asociados a la ELA según el mecanismo biológico en el que participan:

\begin{table}[ht]
	\centering
	\renewcommand{\arraystretch}{1.4}
	\setlength{\tabcolsep}{6pt}
	\adjustbox{center,max width=0.95\textwidth}{
		\begin{tabular}{|p{3.5cm}|p{7.5cm}|p{5.5cm}|}
			\hline
			\textbf{Categoría funcional} & \textbf{Genes asociados} & \textbf{Función principal alterada} \\ 
			\hline
			
			Procesamiento de ARN &
			\textit{TARDBP}, \textit{FUS}, \textit{HNRNPA1}, \textit{HNRNPA2B1}, \textit{MATR3}, \textit{SETX}, \textit{TIA1}, \textit{GLE1} &
			Alteraciones en transcripción, splicing y transporte de ARN. \\ 
			\hline
			
			Autofagia y degradación proteica &
			\textit{SQSTM1}, \textit{OPTN}, \textit{TBK1}, \textit{UBQLN2}, \textit{VCP}, \textit{CCNF}, \textit{CHMP2B}, \textit{CYLD}, \textit{SPG11} &
			Acumulación de proteínas mal plegadas y fallo en los mecanismos de eliminación celular. \\ 
			\hline
			
			Estrés oxidativo y disfunción mitocondrial &
			\textit{SOD1}, \textit{CHCHD10}, \textit{PPARGC1A}, \textit{VAPB}, \textit{SIGMAR1}, \textit{DAO}, \textit{GLT8D1}, \textit{TRPM7} &
			Producción excesiva de especies reactivas de oxígeno y daño mitocondrial. \\ 
			\hline
			
			Metabolismo lipídico y oxidativo &
			\textit{SPTLC1}, \textit{PON1}, \textit{PON2}, \textit{PON3}, \textit{ANG}, \textit{ANXA11} &
			Disregulación del metabolismo lipídico y alteraciones en la homeostasis oxidativa. \\ 
			\hline
			
			Procesos neurodegenerativos &
			\textit{MAPT}, \textit{PSEN1}, \textit{TREM2}, \textit{ALS2} &
			Solapamiento con otras enfermedades neurodegenerativas (ej., Alzheimer o FTD). \\ 
			\hline
			
	\end{tabular}}
	\caption{Principales genes asociados a la ELA, agrupados según su categoría funcional y la función alterada \cite{Hardiman2017,Kiernan2021,Chia2018,Mejzini2019,Renton2014}.}
	\label{tab:genes_introduccion}
\end{table}

La enfermedad se diagnostica mediante criterios clínicos basados en la sintomatología, ya que no existen biomarcadores claros para la enfermedad \cite{Bjornevik2023}. Esta limitación causa un retraso en el diagnóstico y la intervención temprana. Los tratamientos actuales se basan en disminuir la progresión de la enfermedad y atenuar los síntomas lo máximo posible para mejorar la calidad de vida, ya que todavía no existe cura \cite{Lynch2023}. Desde el punto de vista farmacológico, Riluzole es uno de los medicamentos con mayor evidencia de prolongar la supervivencia, aunque su efecto es más bien modesto \cite{RiluzoleCochrane2012, RWE_Riluzole2020}. Además, existen otros fármacos autorizados en ciertos países, como Edaravone, que en poblaciones seleccionadas pueden ralentizar el deterioro funcional \cite{Brooks2024}.

Por lo tanto, este trabajo tiene como objetivo analizar las relaciones entre los genes asociados a la ELA y sus fenotipos correspondientes para identificar patrones que puedan conducir a un análisis estadístico más robusto de la enfermedad. Mediante el uso de técnicas de biología de sistemas, incluyendo las herramientas de construcción y análisis de redes basadas en datos de interacción, se busca comprender mejor los mecanismos moleculares de la enfermedad.

