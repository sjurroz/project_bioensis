\section{Conclusiones}
Este estudio  ha permitido el análisis de redes de interacción proteína-proteína en el contexto de la Esclerosis Lateral Amiotrófica (ELA) a partir de la implementación la biología de sistemas para deconstruir la enfermedad genética del fenotipo . Nuestros resultados demuestran que la integración de datos fenotípicos estandarizados (HPO) con un (\textit{score} 700) y el algoritmo de \textit{Edge Betweenness} constituye la estrategia más adecuada para el análisis del enriquecimiento funcional de las redes PPI compuestas por genes involucrados en nuestro fenotipo.

Desde una perspectiva metodológica, se concluye que:
\begin{itemize}
	\item La selección del umbral es crítica: un \textit{score} de 700 evita tanto la saturación de ruido (falsos positivos) observada en umbrales bajos como la pérdida de información funcional (falsos negativos) de los umbrales estrictos.
	\item La curación manual de genes, aunque es valiosa, presenta un "sesgo de inspección" significativo que puede inducir a errores en algoritmos basados en flujo como \textit{InfoMap}. En contraste, el uso de la ontología HPO demostró ser superior para capturar una red más rica y funcionalmente coherente.
\end{itemize}
