\section{Conclusiones}
Este estudio  ha permitido el análisis de redes de interacción proteína-proteína en el contexto de la Esclerosis Lateral Amiotrófica (ELA) a partir de la implementación la biología de sistemas para deconstruir la enfermedad genética del fenotipo . Nuestros resultados demuestran que la integración de datos fenotípicos estandarizados (HPO) con un (\textit{score} 700) y el algoritmo de \textit{Edge Betweenness} constituye la estrategia más adecuada para el análisis del enriquecimiento funcional de las redes PPI compuestas por genes involucrados en nuestro fenotipo.

Desde una perspectiva metodológica, se concluye que:
\begin{itemize}
	\item La selección del umbral es crítica: un \textit{score} de 700 evita tanto la saturación de ruido (falsos positivos) observada en umbrales bajos como la pérdida de información funcional (falsos negativos) de los umbrales estrictos.
	\item La curación manual de genes, aunque es valiosa, presenta un "sesgo de inspección" significativo que puede inducir a errores en algoritmos basados en flujo como \textit{InfoMap}. En contraste, el uso de la ontología HPO demostró ser superior para capturar una red más rica y funcionalmente coherente.
\end{itemize}

En el ámbito biológico, la validación de la red priorizada confirmó la centralidad de genes canónicos como \textit{SOD1}, \textit{C9orf72}, \textit{TARDBP} y \textit{FUS} dentro del módulo principal, ratificando la capacidad del método para recuperar mecanismos patogénicos conocidos. Asimismo, el análisis topológico permitió discriminar eficazmente genes periféricos como \textbf{PON1} y \textbf{PON3}, cuya desconexión del componente mayoritario sugiere un rol secundario en la red de interacción directa.
Futuras investigaciones deberán centrarse en la validación experimental de las interacciones clave dentro de este módulo, así como en la integración de datos multi-ómicos (transcriptómicos, proteómicos) para modelar la dinámica de esta red a lo largo de la progresión de la enfermedad. En definitiva, este estudio sienta las bases para una exploración más rigurosa y sistémica de la ELA, orientando la búsqueda de dianas terapéuticas hacia los nodos críticos que sostienen esta arquitectura patogénica.

