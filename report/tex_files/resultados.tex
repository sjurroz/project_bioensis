
\section{Resultados}
A partir de la aplicación de la metodología, para el caso de la lista creada a partir del HPO obtuvimos los datos que se muestran en la tabla 3. Mientras que para la lista manual se obtuvieron los datos que se muestran en la tabla 4. 

\begin{table}[h!]
	
	\centering
	
	\renewcommand{\arraystretch}{1.5} % Aumenta el espacio entre filas para que se lea mejor
	
	\setlength{\tabcolsep}{7pt}       % Aumenta el espacio entre columnas
	
	\begin{tabular}{|c|c|c|c|}
		
		\hline
		
		\textbf{ID} & \textbf{\textit{Edge Betweenness}} & \textbf{\textit{Fast Greedy}} & \textbf{\textit{InfoMap}} \\ \hline
		
		\textbf{300} & 12 & 2 & 4 \\ \hline
		
		\textbf{700} & \textbf{24} & 2 & 4 \\ \hline
		
		\textbf{900} & 8 & 6 & 6 \\ \hline
		
	\end{tabular}
	\caption{Resultados de agrupamiento por umbral y algoritmo HPO.}
	
	\label{tab:resultados_cluster}
	
\end{table}

\begin{table}[h!]
	
	\centering
	
	\renewcommand{\arraystretch}{1.5} % Aumenta el espacio entre filas para que se lea mejor
	
	\setlength{\tabcolsep}{7pt}       % Aumenta el espacio entre columnas
	
	\begin{tabular}{|c|c|c|c|}
		
		\hline
		
		\textbf{ID} & \textbf{\textit{Edge Betweenness}} & \textbf{\textit{Fast Greedy}} & \textbf{\textit{InfoMap}} \\ \hline
		
		\textbf{300} & 2 & 2 & 4 \\ \hline
		
		\textbf{700} & \textbf{15} & 2 & 5 \\ \hline
		
		\textbf{900} & 7 & 4 & 4 \\ \hline
		
	\end{tabular}
	\caption{Resultados de agrupamiento por umbral y obtenidos manualmente.}
	
	\label{tab:resultados_cluster}
	
\end{table}

Como se muestra en las tablas, el análisis de las redes mostró un balance óptimo al utilizar el score de \textbf{700}. En contraste el umbral de 300 generó una red con una estructura laxa, mientras que el score 900 resultó estricto, limitando las interacciones. 

En la figura 1.a y 1.b se ilustra el rendimiento de los  métodos de agrupamiento utilizados. En ella se revela que el algoritmo de Edge Betweenness ofreció los resultados más favorables, tanto para la red generada manualmente como para la basada en HPO.
