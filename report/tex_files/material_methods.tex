
\section{Materiales y métodos}
Para este trabajo se realizó un análisis exhaustivo de la red de interacción proteína-proteína (PPI, \textit{protein-protein interaction}) del fenotipo de interés, en donde se utilizaron diferentes herramientas que posibilitaron el estudio de la red. Entre ellas destaca STRING (\textit{Search Tool for the Retrieval of Interacting Genes/protein}), una base de datos biológica y un servidor web de acceso libre enfocado principalmente en las PPIs. Esta herramienta es muy útil para integrar las asociaciones funcionales entre las proteínas de un organismo.

La implementación se realizó en Python, debido a su versatilidad, simplicidad en la sintaxis y amplia disponibilidad de bibliotecas especializadas, que facilitan tanto la comunicación con las APIs de las bases de datos más utilizadas como la representación gráfica de redes complejas. Específicamente, se utilizaron las siguientes librerías:

\begin{itemize}
	\item \texttt{Numpy} (v. 1.26.4): librería esencial para cálculos numéricos y operaciones matriciales.
	
	\item \texttt{Matplotlib} (v. 3.9.2): librería para visualizar datos e imágenes y generar gráficas.
	
	\item \texttt{Pandas} (v. 2.2.3): librería para manipular, limpiar y analizar datos en forma de tablas (\textit{DataFrames}).
	
	\item \texttt{Requests} (v. 2.32.3): librería para realizar solicitudes HTTP y comunicarse con APIs o servicios web.
	
	\item \texttt{Json}: módulo para leer, escribir y procesar datos en formato JSON. Es parte de la biblioteca estándar de Python, por lo que no requiere instalación ni tiene versión propia.
	
	\item \texttt{Igraph} (v 1.0.0): librería para crear, analizar y visualizar redes o grafos, con soporte para métricas y tareas adicionales.
\end{itemize}

El flujo de trabajo diseñado consta de cinco etapas principales que abarcan desde la obtención de genes relacionados con el fenotipo de interés hasta la extracción de conocimiento biológico de la red PPI. Para algunas de las etapas se realizaron múltiples ejecuciones, variando en cada una los parámetros principales. De esta forma, se obtuvieron múltiples combinaciones de resultados, con el objetivo de evaluar qué configuración presentaba los resultados más favorables. Un esquema conceptual del flujo de trabajo puede verse en la \autoref{fig:diagramadeflujobiosis}.

\begin{figure}[h!]
	\centering
	\includegraphics[width=0.7\linewidth]{figures/Diagrama_de_flujo_BioSis.png}
	\caption{Esquema conceptual del flujo de trabajo.}
	\label{fig:diagramadeflujobiosis}
\end{figure}

En primer lugar, se recopiló la lista de genes asociados al fenotipo de interés. Para hacerlo, se adoptaron dos enfoques distintos: por un lado, se utilizaron los genes encontrados tras una revisión bibliográfica manual (\autoref{tab:genes_introduccion}); por el otro, se extrajeron programáticamente a partir de la API de la HPO.

A continuación, mediante la API de STRING, se generó la red PPI para cada una de las dos listas de genes. En este punto se propusieron tres posibles umbrales de puntuación para dar lugar a la red: uno laxo, con un valor de 300; otro intermedio, de 700; y el más estricto, igual a 900. Esto permitió comparar cómo varía la estructura de la red en función del nivel de confianza asignado a las interacciones, evaluando si los patrones observados se mantienen consistentes al aumentar el umbral de evidencia requerido. Una vez obtenidas las seis redes, se representaron gráficamente mediante la librería \texttt{Igraph}. Para cada una de ellas, se llevó a cabo un pequeño análisis exploratorio inicial, en el cual se midieron distintas propiedades básicas de topología como número de nodos, grado, densidad, centralidad, modularidad y dispersión.

Luego, resultó necesario entender los posibles agrupamientos o clusters que tuvo cada red. Este proceso se llevó a cabo con el objetivo de descubrir grupos de genes altamente conectados entre sí, los cuales podrían corresponder a funciones biológicas muy relacionadas con la ELA. Nuevamente, se propusieron dos estrategias diferentes de agrupamiento, que fueron elegidos por su eficacia en conjuntos pequeños de datos (menos de 150 genes) y  su interpretabilidad clara \cite{Hardiman2017}.


\begin{itemize}
	\item\textbf{\textit{Clustering} basado en la centralidad de intermediación de enlace (\textit{Edge Betweenness Clustering})}: es un método de detección de comunidades propuesto por Girvan y Newman en el año 2002 \cite{newman2004modularity} Este algoritmo consiste en calcular la centralidad de intermediación para todas las aristas donde se eliminan aquellas con los valores más altos, dicho proceso se hace de manera iterativa con el objetivo de separar la red en subconjuntos mayormente conectados de forma interna.
	
	Este proceso continúa hasta obtener la red con agrupamientos que presenten el valor más alto de la modularidad, la cual cuantifica la calidad de la conexión interna de los agrupamientos respecto al resto de la red.. Dicha métrica tiene un intervalo [-1,1] donde el valor más cercano o igual a 1 determina que el  agrupamiento está bien definido (\autoref{tab:valores_modularidad}). 
	
	\begin{table}[ht]
		\centering
		\renewcommand{\arraystretch}{1.4}
		\setlength{\tabcolsep}{6pt}
		\begin{tabular}{|p{3cm}|p{10cm}|}
			\hline
			\textbf{Valor de $Q$} & \textbf{Interpretación} \\
			\hline
			$Q \approx 1$ & Agrupamiento fuerte y bien definido. \\
			\hline
			$0 < Q < 1$ & Partición clara, aunque no completamente aislada, ya que existe cierta conectividad entre los agrupamientos. \\
			\hline
			$Q \approx 0$ & Estructura compuesta principalmente por conexiones aleatorias. \\
			\hline
			$-1 < Q < 0$ & La partición es peor que el azar, con más conexiones entre comunidades que dentro de ellas. \\
			\hline
			$Q \approx -1$ & Agrupamiento totalmente incoherente. \\
			\hline
		\end{tabular}
		\caption{\textbf{Significado de los valores de la modularidad (Q).}}
		\label{tab:valores_modularidad}
	\end{table}
	
	\item \textbf{\textit{Clustering} basado en el algoritmo InfoMap}: es un método de detección de comunidades propuesto por Martin Rosvall y Carl T. Bergstrom en el año 2008 \cite{rosvall2008infomap}, que no depende de una métrica estructural predeterminada y detecta comunidades de tamaño variable con submódulos jerárquicos. Este algoritmo detecta los agrupamientos analizando el flujo de información entre nodos usando caminatas aleatorias que simulan el movimiento dentro de una red y eligen el próximo nodo al azar.
	
	Las múltiples caminatas se utilizan para calcular la cantidad de bits que se necesitan para describir el recorrido: si el recorrido pasó una gran parte dentro de un mismo grupo de nodos se considera una comunidad densamente conectada internamente; en el caso contrario, se considera que hay una conexión más débil entre comunidades.  
\end{itemize}


Habiendo implementado ambos métodos, se obtuvieron los mejores \textit{clusters} propuestos por cada uno. Es decir, se generaron doce resultados en total: dos conjuntos de \textit{clusters} —uno por cada algoritmo— para cada una de las seis redes analizadas.

Para el estudio exhaustivo de los resultados, se recurrió a dos enfoques complementarios para analizar la coherencia biológica de los mismos. Por una parte, se utilizó una validación técnica, basada en principios de topología de redes (calculando las métricas ya mencionadas como densidad, centralidad, modularidad, etc), pero teniendo en cuenta el número de \textit{clusters}. Por otro lado, se consideró también el significado biológico, mediante un análisis de enriquecimiento funcional de sobrerrepresentación (ORA, \textit{Over-Representation Analysis}). Para cada uno de los doce resultados del clustering, se llevaron a cabo múltiples ejecuciones del ORA, fijando el \textit{p}-valor y variando la base de datos de referencia, con el fin de comparar la consistencia de los resultados entre distintas fuentes de anotación funcional. Se utilizaron tres de las más habituales: Gene Ontology (GO), Kyoto Encyclopedia of Genes and Genomes (KEGG) y Reactome. 

Finalmente, se examinó el equilibrio entre el número de \textit{clusters}, la cantidad total de funciones enriquecidas y la media de funciones por \textit{cluster}, buscando la configuración con el mejor compromiso entre estructura de la red y coherencia funcional.


